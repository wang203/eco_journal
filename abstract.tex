\begin{abstract}
\hfill \break


\subsection*{workshop}


The billions of public photos on online social media sites contain a
vast amount of latent visual information about the world.  In this
paper, we study the feasibility of observing the state of the natural
world by recognizing specific types of scenes and objects in
large-scale social image collections. More specifically, we study
whether we can recreate satellite maps of snowfall by automatically
recognizing snowy scenes in geo-tagged, timestamped images from
Flickr.  Snow recognition turns out to be a surprisingly difficult and
under-studied problem, so we test a variety of modern scene
recognition techniques on this problem and introduce a large-scale,
realistic dataset of images with ground truth annotations.  As an
additional proof-of-concept, we test the ability of recognition
algorithms to detect a particular species of flower, the California
Poppy, which could be used to give biologists a new source of data on
its geospatial distribution over time.


\hfill \break
\hfill \break
===================================
\hfill \break
\hfill \break
\subsection*{www}

The popularity of social media websites like Flickr
and Twitter has created enormous collections of user-generated content
online. Latent in these content collections are observations of the
world: each photo is a visual snapshot of what the world looked like
at a particular point in time and space, for example, while each tweet
is a textual expression of the state of a person and his or her
environment. Aggregating these observations across millions
of social sharing users could lead to new techniques for large-scale
monitoring of the state of the world and how it is changing over
time. In this paper we step towards that goal, showing that by
analyzing the tags and image features of geo-tagged, time-stamped
photos we can measure and quantify the occurrence of ecological
phenomena including ground snow cover, snow fall and vegetation
density.  We compare several techniques for dealing with the large
degree of noise in the dataset, and show how machine learning can be
used to reduce errors caused by misleading tags and ambiguous visual
content. We evaluate the accuracy of these techniques by comparing to
ground truth data collected both by surface stations and by
Earth-observing satellites. Besides the immediate application to
ecology, our study gives insight into how to accurately crowd-source                                                                                                                           
other types of information from large, noisy social sharing datasets.

\end{abstract}
