\section{related work}
\label{sec:relatedwork}


A variety of recent work has studied how to apply computational
techniques to analyze online social datasets in order to aid research
in many disciplines Lazer\etal~\cite{lazer09}. Much of this work has studied
questions in sociology and human interaction, such as how friendships
form Crandall\etal~\cite{feedback08kdd}, how information flows through social
networks Liben-Nowell and Kleinberg~\cite{libennowell08}, how people move through
space Brockmann\etal~\cite{brockmann06}, and how people influence their
peers Anagnostopoulos\etal~\cite{anagnostpopoulos08}.  The goal of these projects is not to
measure data about the physical world itself, but instead to discover
interesting properties of human behavior using social networking sites
as a convenient data source.

\xhdr{Crowd-sourced observational data.}
Other studies have shown the power of social networking sites as a
source of observational data about the world itself.  Bollen\etal~\cite{bollen11twitter} use data from Twitter to try to measure
the aggregated emotional state of humanity, computing mood across six
dimensions according to a standard psychological
test. Intriguingly, they find that these changing mood states
correlate well with the Dow Jones Industrial Average, allowing stock
market moves to be predicted up to 3 days in advance.  However their
test dataset is relatively small, consisting of only three weeks of
trading data.  Like us, Jin\etal~\cite{jin10prediction} use
Flickr as a source of data for prediction, but they estimate the
adoption rate of consumer photos by monitoring the frequency of tag
use over time. They find that the volume of Flickr tags is 
correlated  with with sales of two products, Macs and iPods. They also
estimate geo-temporal distributions of these sales over time but do
not compare to ground truth, so it is unclear how accurate these
estimates are. In contrast, we evaluate our techniques against a large
ground truth dataset, where the task is to accurately predict the
distribution of a phenomenon (e.g. snow) across an entire continent 
each day for several years.


\xhdr{Crowd-sourcing from social media.} 
Several recent studies have shown the power of social media  for observing
the world itself, as a special case of `social sensing' Aggarwal\etal~\cite{Aggarwal:2013vh}.
This work includes using Twitter data to measure collective emotional
state Golder\etal~\cite{Golder:2011cy} (which, in turn, has found to
be predictive of stock moves Bollen\etal~\cite{bollen11twitter}),
predicting product adoption rates and political election
outcomes Jin\etal~\cite{jin10prediction}, and collecting data about
earthquakes and other natural disasters Sakaki\etal~\cite{Sakaki:2010uv}.
%Others have tracked the geo-temporal distributions of social media
%tags, query search terms, and English words in large
%datasets~\cite{singh10socialpixels,
%vlachos2004identifying,chien2005semantic,Backstrom:2008tv,Vadrevu2008,ginsberg09flu,Sadilek:2012wp,Cook:2012wj,Michel:2011gm,radinsky2011word,Radinsky:2012ta};

Particularly striking examples include Ginsberg
\textit{et al}~\cite{ginsberg09flu}, who show that 
geo-temporal properties of web search queries can
predict the spread of flu, and Sadilek \textit{et al}~\cite{Sadilek:2012wp} who show that
Twitter feeds can predict when a given person will
fall ill.


\xhdr{Crowd-sourced geo-temporal data.}
Other work has used online data to predict geo-temporal distributions,
but again in domains other than ecology.  Perhaps the most
striking is the work of Ginsberg \textit{et al}~\cite{ginsberg09flu},
who show that by monitoring the geospatial distribution of search
engine queries related to flu symptoms, the spread of the H1N1
flu can be estimated several days before the official statistics produced by traditional
means.
DeLongueville \textit{etal}~\cite{delongueville09} study tweets related to a major fire in
France, but their analysis is at a very small scale (a few dozen
tweets) and their focus is more on human reactions to the fire as
opposed to using these tweets to estimate the fire's position and
severity.  In perhaps the most related existing work to ours,
 Singh \textit{et al}~\cite{singh10socialpixels} create
geospatial heat maps (dubbed ``social pixels'') of various
tags, including snow and greenery, but their focus is on developing a
formal database-style algebra for describing queries on these systems
and for creating visualizations. They do not consider how to produce
accurate predictions from these visualizations, nor do they compare to
any ground truth.

\xhdr{Accuracy of geo and temporal data on Flickr.}
Over a sample of 10 million images on Flickr.com, 37\% (only a small subset) of them probably have incorrect timestamp Thomee\etal~\cite{thomee2014s}.
The accuracy of geo-location is limited due to the camera device, and GPS precision.

Meanwhile, a lot of works are trying to correct estimate or correct geo-location of Flickr images.
%~\cite{Geo-Clustering of Images With Missing GeoTags} 
Singh\etal~\cite{singh2010geo}estimates where images are taken for those missing geo-tags. They are optimizing a graph clustering problem. Attributes in their graph include textual tags, timestamps and vision content. It's inspired by an earlier work Crandall\etal~\cite{crandall2009mapping}.
%~\cite{Geo-Location Estimation of Flickr Images: SocialWeb Based Enrichment} 


Timestamp is harder to be accurate due to the same reason of geo-tags and the time-zone problem. 
%In paper %~\cite{Who's Time is it anyway?}, 
Thomee\etal~\cite{thomee2014s}  show  a detailed analysis in disagreement of camera time and GPS time. They also estimate a more accurate timestamp when users taking multiple images in a short timespan.
Hauff\etal~\cite{hauff2012geo} consider textual meta-data to correct geo tags. 
They also found for users active on both Flickr and Twitter, the Twitter post at around the same time the images are taken can be a reliable reference to estimate the approximate location.



%------------------------------------------------------------


The specific application we consider here is inferring
information about the state of the natural world from social media.
Existing work has analyzed textual content, including text tags and
Twitter feeds, in order to do this. Hyvarinen and
Saltikoff~\cite{meteo} use tag search on Flickr to validate
metereological satellite observations, although the analysis is done
by hand. Zhang \textit{et al}~\cite{ecology2012www} take a large
collection of geo-tagged and time-stamped Flickr photos and search for
snow-related tags to produce estimates of geo-temporal snowfall
distributions, and evaluate them against satellite snow maps.
Singh \textit{et al}~\cite{singh10socialpixels} visualize geospatial
distributions of photos tagged ``snow'' as an example of
their Social Pixels framework, but they study the
database theory needed to perform this analysis
and do not consider the prediction problem.  


Few papers have used actual image content
analysis as we do here. Leung and Newsam~\cite{Leung:2010wa} use scene
analysis in geo-tagged photos to infer land cover and land use types.
Murdock \textit{et al}~\cite{murdock} analyze  geo-referenced
stationary webcam feeds to estimate cloud cover on a day-by-day
basis, and then use these estimates to recreate satellite cloud cover
maps.  Webcams offer a complimentary data source to the social media
images we consider here: on one hand, analyzing webcam data is made
easier by the fact that the camera is stationary and offers dense
temporal resolution; on the other hand, their observations are restricted to 
where public webcams exist, whereas photos on social media sites 
offer a potentially much denser spatial sampling of the world.

We note that these applications are related to citizen science
projects where volunteers across a wide geographic area send 
in observations~\cite{greatsunflower}, Fink\etal~\cite{ebirds}, King\etal~\cite{king09snowtweets}. These projects
often use social media, but require observations to be made
explicitly, whereas in our work we ``passively'' analyze social
media feeds generated by untrained and unwitting individuals.

\xhdr{Detecting snow in images} We know of only a handful of
papers that have explicitly considered snow detection in
images. Perhaps the most relevant is the 2003 work of Singhal
\textit{et al}~\cite{singhal2003spatialcontext}, Luo\etal~\cite{luo2003spatialcontext}
which studies this in the context of detecting ``materials'' like
water, grass, sky, etc. They calculate local color and texture
features at each pixel, and then compute a probability
distribution over the materials at each pixel using a neural
network. They partition the image into segments by thresholding these
belief values, and assign a label to each segment with a probabilistic
framework that considers both the beliefs and simple
contextual information like relative location.
 They find that sky and grass
are relatively easy to classify, while snow and water are
most difficult.  Follow-up work Boutell\etal~\cite{boutell2006semanticfeature}, Boutell~\cite{boutell2005exploiting}
applied more modern techniques like
support vector machines.  Barnum \textit{et al}~\cite{rain2009IJCV}
detect falling snow and rain, a complementary problem
to the one we study here of detecting fallen
snow.

Papers in the scene recognition literature have considered
snowy scenes amongst their scene categories; for instance, Li \textit{et
  al}~\cite{li2009totalscene,li2007event} mention snow as one possible
component of their scene parsing framework, but do not present
experimental results. The SUN database of Xiao \textit{et
  al}~\cite{XiaoHEOT10} includes several snow-related classes like
``snowfield,'' ``ski slope,'' ``ice shelf,'' and ``mountain snowy,''
but other categories like ``residential neighborhood'' sometimes have
snow and sometimes do not, such that detecting these scenes alone
is not sufficient for our purposes. 






\xhdr{Vegetation classification.}
%http://www.google.com/url?sa=t&rct=j&q=&esrc=s&source=web&cd=5&ved=0CEkQFjAE&url=http%3A%2F%2Fhomes.cs.washington.edu%2F~neeraj%2Fpapers%2Fnk_eccv2012_leafsnap.pdf&ei=W9n8U5z7JoOTyATZloLICA&usg=AFQjCNHzDMozsenIFTdGySrMk0Vw8nhLhA&sig2=CPxNj5C0S-b3YDpR7-jPHg&bvm=bv.73612305,d.aWw

%~\cite{Leafsnap: A Computer Vision System for Automatic Plant Species Identification} 

Kumar\etal~\cite{kumar2012leafsnap}identifies plant species by leaf images. They focus on accurate leaf segmentation according to color difference of leaf and background, curvature distribution over scale, and nearest neighbor matching.

%~\cite{Comparisons of Gist Models in Rapid Scene categorization tasks}

Siagian\etal~\cite{siagian2008comparison}introduces multiple Gist models in scene classification. There happen to be a test set of vegetation shows Gist feature works great on vegetation classification.

%~\cite{Detecting subpixel woody vegetation in digital imagery using two artificial intelligence approaches}

Foschi\etal~\cite{foschi1997detecting} is using man-made mask and neural networks.
(could be compared with deep learning)

%a lot of works use remote sensing 

%http://www.google.com/url?sa=t&rct=j&q=&esrc=s&source=web&cd=3&ved=0CDMQFjAC&url=http%3A%2F%2Fwww.researchgate.net%2Fpublication%2F4309074_Greenery_Image_and_Non-greenery_Image_Classification_Using_Adaptive_Neuro-Fuzzy_Inference_System%2Flinks%2F09e4150ddce7a3514d000000&ei=Edr8U6yzJMOAygSg4YGAAw&usg=AFQjCNGXbvsw1sd1aZA1vAnk_f-4n5bwGA&sig2=62HbvxAtNQXCkCH_raYdig&bvm=bv.73612305,d.aWw
This paper Balamurugan\etal~\cite{balamurugan2007greenery}
%~\cite{greenery and non-greenery image classification using adaptive neuro-guzzy inference system}
is the closest one to our purpose.
They consider
color, texture features in images and get good result. But they only test on a very limited dataset where the positive images are either with one tree in the center or full of trees or meadow. This is inadequate when we are working with   very large number of public shared images.
%200 imgs (50*50)


\xhdr{Citizen science.}
While some volunteer-based biology efforts like the Lost Ladybug
Project~\cite{lostladybug} and the Great Sunflower
Project~\cite{greatsunflower} use social networking sites to
organize and recruit volunteer observers, we are not aware of any
work that has attempted to passively mine ecological data from social media
sites. The visual data in online social networking sites provide a
unique resource for tracking biological phenomena:  because they are
images, this data can be verified in ways that simple text 
cannot.  In addition, the rapidly expanding quantity
of online images with geo-spatial and temporal metadata creates a
fine-scale record of what is happening across the globe.  However, to
unlock the latent information in these vast photo collections, we need
 mining and recognition tools that can efficiently
process large numbers of images, and robust statistical models that
can handle incomplete and incorrect observations.


